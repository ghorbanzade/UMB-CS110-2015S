%%%%%%%%%%%%%%%%%%%%%%%%%%%%%%%%%%%%%%%%%%%%%%%%%%%%%%%%%%%%%%%%%%%%%%
% UMB-CS110-2015S: Introduction to Computing
% Copyright 2015 Pejman Ghorbanzade <pejman@ghorbanzade.com>
% Creative Commons Attribution-ShareAlike 4.0 International License
% More info: https://github.com/ghorbanzade/UMB-CS110-2015S
%%%%%%%%%%%%%%%%%%%%%%%%%%%%%%%%%%%%%%%%%%%%%%%%%%%%%%%%%%%%%%%%%%%%%%

\def \topDirectory {.}
\def \texDirectory {\topDirectory/src/main/tex}

\documentclass[12pt,letterpaper,twoside]{article}
\usepackage{\texDirectory/template/style/directives}
\usepackage{\texDirectory/template/style/assignment}
%%%%%%%%%%%%%%%%%%%%%%%%%%%%%%%%%%%%%%%%%%%%%%%%%%%%%%%%%%%%%%%%%%%%%%%%%%%%%%
% CS110: Introduction to Computing
% Copyright 2015 Pejman Ghorbanzade <mail@ghorbanzade.com>
% Creative Commons Attribution-ShareAlike 4.0 International License
% https://github.com/ghorbanzade/UMB-CS110-2015S/blob/master/LICENSE
%%%%%%%%%%%%%%%%%%%%%%%%%%%%%%%%%%%%%%%%%%%%%%%%%%%%%%%%%%%%%%%%%%%%%%%%%%%%%%

\course{id}{CS110}
\course{name}{Introduction to Computing}
\course{venue}{Tue/Thu, 5:30 PM - 6:45 PM}
\course{semester}{Spring 2015}
\course{department}{Department of Computer Science}
\course{university}{University of Massachusetts Boston}

\instructor{name}{Pejman Ghorbanzade}
\instructor{title}{}
\instructor{position}{Student Instructor}
\instructor{email}{pejman@cs.umb.edu}
\instructor{phone}{617-287-6419}
\instructor{office}{S-3-124B}
\instructor{office-hours}{Tue/Thu 19:00-20:30}
\instructor{address}{University of Massachusetts Boston, 100 Morrissey Blvd., Boston, MA}


\begin{document}

\doc{title}{Lab Session Problems}
\doc{points}{0}

\prepare{header}

\subsection*{Week 6}
\hfill \textbf{March 12, 2015}

Write a class \texttt{Cat.java} with two instance variables position and name and a method \texttt{move(int dist}). Method \texttt{move(int dist)} changes position of the object on which it is invoked and at every call will print a message of the following form. At this point, you are not required to use constructors, accessors or modifiers.

\begin{verbatim}
object.name moved dist forward.
Current posiion of object.name: object.position
\end{verbatim}

Use your class \texttt{Cat.java} in another class \texttt{Cat1Dtest.java} following the steps below.

\begin{enumerate}[itemsep=0pt]
\item Create a cat object and name it kitty.
\item Initialize position of kitty to 0 of the x-axis.
\item Use the move method as you like to change position of kitty.
\item Modify your class \texttt{Cat.java} so that you can use it in another class \texttt{Cat2Dtest.java} where you can control position of kitty in Cartesian Coordinates.
\end{enumerate}

\end{document}
