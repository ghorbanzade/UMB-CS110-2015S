%%%%%%%%%%%%%%%%%%%%%%%%%%%%%%%%%%%%%%%%%%%%%%%%%%%%%%%%%%%%%%%%%%%%%%%%%%%%%%
% CS110: Introduction to Computing
% Copyright 2015 Pejman Ghorbanzade <mail@ghorbanzade.com>
% Creative Commons Attribution-ShareAlike 4.0 International License
% https://github.com/ghorbanzade/UMB-CS110-2015S/blob/master/LICENSE
%%%%%%%%%%%%%%%%%%%%%%%%%%%%%%%%%%%%%%%%%%%%%%%%%%%%%%%%%%%%%%%%%%%%%%%%%%%%%%

\def \topDirectory {.}
\def \texDirectory {\topDirectory/src/main/tex}

\documentclass[12pt,letterpaper,twoside]{article}
\usepackage{\texDirectory/template/style/directives}
\usepackage{\texDirectory/template/style/assignment}
%%%%%%%%%%%%%%%%%%%%%%%%%%%%%%%%%%%%%%%%%%%%%%%%%%%%%%%%%%%%%%%%%%%%%%%%%%%%%%
% CS110: Introduction to Computing
% Copyright 2015 Pejman Ghorbanzade <mail@ghorbanzade.com>
% Creative Commons Attribution-ShareAlike 4.0 International License
% https://github.com/ghorbanzade/UMB-CS110-2015S/blob/master/LICENSE
%%%%%%%%%%%%%%%%%%%%%%%%%%%%%%%%%%%%%%%%%%%%%%%%%%%%%%%%%%%%%%%%%%%%%%%%%%%%%%

\course{id}{CS110}
\course{name}{Introduction to Computing}
\course{venue}{Tue/Thu, 5:30 PM - 6:45 PM}
\course{semester}{Spring 2015}
\course{department}{Department of Computer Science}
\course{university}{University of Massachusetts Boston}

\instructor{name}{Pejman Ghorbanzade}
\instructor{title}{}
\instructor{position}{Student Instructor}
\instructor{email}{pejman@cs.umb.edu}
\instructor{phone}{617-287-6419}
\instructor{office}{S-3-124B}
\instructor{office-hours}{Tue/Thu 19:00-20:30}
\instructor{address}{University of Massachusetts Boston, 100 Morrissey Blvd., Boston, MA}


\begin{document}

\doc{title}{Lab Session Problems}
\doc{points}{0}

\prepare{header}

\subsection*{Week 12}
\hfill \textbf{May 07, 2015}

Write a recursive method \texttt{isAchievable()} with following method signature that returns true if there is a group of elements in the array \texttt{array} such that sum of its values is equal to the value of parameter \texttt{target}.

\lstset{language=java,tabsize=2,numbers=none}
\begin{lstlisting}
private static boolean isAchievable(int startIndex, double[] array, double target);
\end{lstlisting}

Table \ref{tab2} represents output of the method for different sets of parameters.

\begin{table}[H]\centering
\begin{tabular}{cccc}
\hline
start & array & target & output\\
\hline
$0$ & $\{1, 2, 8\}$ & $11$ & true\\
$0$ & $\{1, 2, 8\}$ & $9$  & true\\
$1$ & $\{1, 2, 8\}$ & $9$  & false\\
$0$ & $\{1, 2, 8\}$ & $5$  & false\\
$0$ & $\{1\}$       & $1$  & true\\
$0$ & $\{\} $       & $0$  & true\\
\hline
\end{tabular}
\caption{Output of method \texttt{isAchievable()} for different parameter lists}\label{tab2}
\end{table}

For simplicity, suppose the number of elements in the set \texttt{array} is no more than 1000.

\end{document}
