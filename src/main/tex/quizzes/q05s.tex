%%%%%%%%%%%%%%%%%%%%%%%%%%%%%%%%%%%%%%%%%%%%%%%%%%%%%%%%%%%%%%%%%%%%%%%%%%%%%%
% CS110: Introduction to Computing
% Copyright 2015 Pejman Ghorbanzade <mail@ghorbanzade.com>
% Creative Commons Attribution-ShareAlike 4.0 International License
% https://github.com/ghorbanzade/UMB-CS110-2015S/blob/master/LICENSE
%%%%%%%%%%%%%%%%%%%%%%%%%%%%%%%%%%%%%%%%%%%%%%%%%%%%%%%%%%%%%%%%%%%%%%%%%%%%%%

\def \topDirectory {.}
\def \texDirectory {\topDirectory/src/main/tex}

\documentclass[12pt,letterpaper,twoside]{article}
\usepackage{\texDirectory/template/style/directives}
\usepackage{\texDirectory/template/style/assignment}
%%%%%%%%%%%%%%%%%%%%%%%%%%%%%%%%%%%%%%%%%%%%%%%%%%%%%%%%%%%%%%%%%%%%%%%%%%%%%%
% CS110: Introduction to Computing
% Copyright 2015 Pejman Ghorbanzade <mail@ghorbanzade.com>
% Creative Commons Attribution-ShareAlike 4.0 International License
% https://github.com/ghorbanzade/UMB-CS110-2015S/blob/master/LICENSE
%%%%%%%%%%%%%%%%%%%%%%%%%%%%%%%%%%%%%%%%%%%%%%%%%%%%%%%%%%%%%%%%%%%%%%%%%%%%%%

\course{id}{CS110}
\course{name}{Introduction to Computing}
\course{venue}{Tue/Thu, 5:30 PM - 6:45 PM}
\course{semester}{Spring 2015}
\course{department}{Department of Computer Science}
\course{university}{University of Massachusetts Boston}

\instructor{name}{Pejman Ghorbanzade}
\instructor{title}{}
\instructor{position}{Student Instructor}
\instructor{email}{pejman@cs.umb.edu}
\instructor{phone}{617-287-6419}
\instructor{office}{S-3-124B}
\instructor{office-hours}{Tue/Thu 19:00-20:30}
\instructor{address}{University of Massachusetts Boston, 100 Morrissey Blvd., Boston, MA}


\begin{document}

\doc{title}{Solution to Quiz 5}
\doc{date-pub}{May 05, 2015 at 01:00 PM}
\doc{date-due}{May 08, 2015 at 11:00 PM}
\doc{points}{4}

\prepare{header}

\section*{Question 1}

Write a program \texttt{QuadraticSolver.java} that asks for three real numbers $a$, $b$ and $c$ from the user and uses a method \texttt{findSolution()} with given method signature to solve for $x$ with three digits of precision, the quadratic expression shown in Equation \ref{eq2}.
\begin{equation}
ax^2+bx+c=0
\label{eq2}
\end{equation}

\lstset{language=java,tabsize=2,numbers=none}
\begin{lstlisting}
private static double[] findSolution(double[] coefficients) throws NegativeDiscriminantException, ArithmeticException;
\end{lstlisting}

Your program is expected to maintain its control flow regardless of the user entries. User should be asked for inputs as long as his previous entries are not acceptable or result in a negative discriminant.

\subsection*{Solution}

\begin{enumerate}
\item Class \texttt{NegativeDiscriminantException.java}
\lstset{language=java, tabsize=2, numbers=left}
\begin{lstlisting}
// As there is no built-in exception to be thrown
// when discriminant of a polynomial expression is negative
// we should build our own exception
// and make sure it extends from class Exceptions
public class NegativeDiscriminantException extends Exception {
	// a simple attribute to show a message to user when exception happens
	private String message;
	// a public constructor to allow instantiation of exceptions
	public NegativeDiscriminantException(String message) {
		this.message = message;
	}
	// a simple getter to override the getMessage() method
	// which comes with class Exception
	public String getMessage() {
		return message;
	}
}
\end{lstlisting}
\item Class \texttt{QuadraticSolver.java}
\begin{lstlisting}
import java.util.InputMismatchException;
import java.util.Scanner;
public class QuadraticSolver {
	private static double[] findSolution(double[] coefficients) throws NegativeDiscriminantException, ArithmeticException {
		// compute discriminant (this is safe)
		double discriminant = Math.pow(coefficients[1],2)-(4*coefficients[0]*coefficients[2]);
		// throw Exception when discriminant is negative
		if (discriminant < 0)
			throw new NegativeDiscriminantException("Discriminant is Negative");
		// throw Exception when coefficient a is zero
		else if (coefficients[0] == 0) {
			throw new ArithmeticException("Coefficient 'a' cannot be zero.");
		}
		// If we have only one solution
		else if (discriminant == 0) {
			double[] output = new double[1];
			output[0] = -coefficients[1]/(2*coefficients[0]);
			return output;
		}
		// If we have two solutions
		else {
			double[] output = new double[2];
			output[0] = ( -coefficients[1] + Math.sqrt(discriminant) )/( 2*coefficients[0] );
			output[1] = ( -coefficients[1] - Math.sqrt(discriminant) )/( 2*coefficients[0] );
			return output;
		}
	} // end of findSolution() method
	public static void main(String[] args) {
		// a Scanner object to read user input
		Scanner input = new Scanner(System.in);
		// an array of size three for coefficients
		double[] params = new double[3];
		// an uninitialized array to store solutions given by findSolution method
		double[] solutions;
		// use a while loop to repeat asking for numbers in case required
		while (true) {
			// surround code with try block to handle possible exceptions
			try {
				// ask for three coefficients a, b and c
				for (int i = 0; i < params.length; i++) {
					System.out.printf("Enter Coefficient %d: ",i);
					// ask the user to input double
					// this might throw InputMismatchException
					params[i] = input.nextDouble();
				}
				// find solutions for expression formed based on user input
				// this might throw NegativeDiscriminantException or ArithmeticException
				solutions = findSolution(params);
				// if control flow reaches this part, it means no exception has happened we're good to go out of while loop
				break;
			}
			// Handle InputMismatchException
			catch (InputMismatchException e) {
				// clear previous tokens
				input.next();
				// for InputMismatchException, we cannot use e.getMessage()
				System.out.println("Please enter coefficients of type double");
				System.out.println("Please Retry.\n");
			}
			// Handle Exceptions thrown by method findSolution()
			catch (NegativeDiscriminantException|ArithmeticException e) {
				System.out.println(e.getMessage());
				System.out.println("Please Retry.\n");
			}
		}
		// close the Scanner object
		input.close();
		// print the solutions with three digits of precision
		System.out.println("Solutions are: ");
		for (int i = 0; i < solutions.length; i++)
			System.out.printf("%.3f\n",solutions[i]);
	}
}
\end{lstlisting}
\end{enumerate}

\end{document}
