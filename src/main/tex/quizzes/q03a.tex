%%%%%%%%%%%%%%%%%%%%%%%%%%%%%%%%%%%%%%%%%%%%%%%%%%%%%%%%%%%%%%%%%%%%%%
% UMB-CS110-2015S: Introduction to Computing
% Copyright 2015 Pejman Ghorbanzade <pejman@ghorbanzade.com>
% Creative Commons Attribution-ShareAlike 4.0 International License
% More info: https://github.com/ghorbanzade/UMB-CS110-2015S
%%%%%%%%%%%%%%%%%%%%%%%%%%%%%%%%%%%%%%%%%%%%%%%%%%%%%%%%%%%%%%%%%%%%%%

\def \topDirectory {.}
\def \texDirectory {\topDirectory/src/main/tex}

\documentclass[12pt,letterpaper,twoside]{article}
\usepackage{\texDirectory/template/style/directives}
\usepackage{\texDirectory/template/style/assignment}
%%%%%%%%%%%%%%%%%%%%%%%%%%%%%%%%%%%%%%%%%%%%%%%%%%%%%%%%%%%%%%%%%%%%%%%%%%%%%%
% CS110: Introduction to Computing
% Copyright 2015 Pejman Ghorbanzade <mail@ghorbanzade.com>
% Creative Commons Attribution-ShareAlike 4.0 International License
% https://github.com/ghorbanzade/UMB-CS110-2015S/blob/master/LICENSE
%%%%%%%%%%%%%%%%%%%%%%%%%%%%%%%%%%%%%%%%%%%%%%%%%%%%%%%%%%%%%%%%%%%%%%%%%%%%%%

\course{id}{CS110}
\course{name}{Introduction to Computing}
\course{venue}{Tue/Thu, 5:30 PM - 6:45 PM}
\course{semester}{Spring 2015}
\course{department}{Department of Computer Science}
\course{university}{University of Massachusetts Boston}

\instructor{name}{Pejman Ghorbanzade}
\instructor{title}{}
\instructor{position}{Student Instructor}
\instructor{email}{pejman@cs.umb.edu}
\instructor{phone}{617-287-6419}
\instructor{office}{S-3-124B}
\instructor{office-hours}{Tue/Thu 19:00-20:30}
\instructor{address}{University of Massachusetts Boston, 100 Morrissey Blvd., Boston, MA}


\begin{document}

\doc{title}{Quiz 3(a)}
\doc{date-pub}{Mar 31, 2015 at 01:00 PM}
\doc{date-due}{Apr 01, 2015 at 11:00 PM}
\doc{points}{4}

\prepare{header}

\section*{Question 1}

You are asked to develop a simple ticket printing machine for box-office of a small cinema with a total of 50 seats, 10 seats in each row.
Ticketing procedure is based on a few simple policies as follows:

\begin{enumerate}[itemsep=-2mm,label={}]
  \item Rows closer to screen are always prioritized.
  \item Seats in a row are filled from the leftmost to the rightmost.
  \item All seats assigned to a group of people must be in the same row.
\end{enumerate}

Develop a class \texttt{Theater.java} with class members that you think are required for this objective and use your class in a program \texttt{TicketSale.java} that prompts ticket seller for number of tickets to print, showing the number of total available seats at the same time.
If sufficient number of seats are available in a single row, your program should print the list of seats assigned to the group.
Otherwise, it should show a warning indicating tickets are not available for the group.
Program terminates when all tickets are sold and no more seats are available.

\prepare{footer}

\end{document}
