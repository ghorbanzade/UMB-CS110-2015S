%%%%%%%%%%%%%%%%%%%%%%%%%%%%%%%%%%%%%%%%%%%%%%%%%%%%%%%%%%%%%%%%%%%%%%%%%%%%%%
% CS110: Introduction to Computing
% Copyright 2015 Pejman Ghorbanzade <mail@ghorbanzade.com>
% Creative Commons Attribution-ShareAlike 4.0 International License
% https://github.com/ghorbanzade/UMB-CS110-2015S/blob/master/LICENSE
%%%%%%%%%%%%%%%%%%%%%%%%%%%%%%%%%%%%%%%%%%%%%%%%%%%%%%%%%%%%%%%%%%%%%%%%%%%%%%

\def \topDirectory {../..}
\def \texDirectory {\topDirectory/src/main/tex}

\documentclass[10pt, compress]{beamer}

\usepackage{\texDirectory/template/style/directives}
%%%%%%%%%%%%%%%%%%%%%%%%%%%%%%%%%%%%%%%%%%%%%%%%%%%%%%%%%%%%%%%%%%%%%%%%%%%%%%
% CS110: Introduction to Computing
% Copyright 2015 Pejman Ghorbanzade <mail@ghorbanzade.com>
% Creative Commons Attribution-ShareAlike 4.0 International License
% https://github.com/ghorbanzade/UMB-CS110-2015S/blob/master/LICENSE
%%%%%%%%%%%%%%%%%%%%%%%%%%%%%%%%%%%%%%%%%%%%%%%%%%%%%%%%%%%%%%%%%%%%%%%%%%%%%%

\course{id}{CS110}
\course{name}{Introduction to Computing}
\course{venue}{Tue/Thu, 5:30 PM - 6:45 PM}
\course{semester}{Spring 2015}
\course{department}{Department of Computer Science}
\course{university}{University of Massachusetts Boston}

\instructor{name}{Pejman Ghorbanzade}
\instructor{title}{}
\instructor{position}{Student Instructor}
\instructor{email}{pejman@cs.umb.edu}
\instructor{phone}{617-287-6419}
\instructor{office}{S-3-124B}
\instructor{office-hours}{Tue/Thu 19:00-20:30}
\instructor{address}{University of Massachusetts Boston, 100 Morrissey Blvd., Boston, MA}

\usepackage{\texDirectory/template/style/beamerthemeUmassLecture}
\usepackage[school]{\texDirectory/template/pgf-umlcd/pgf-umlcd}
\doc{number}{12}
%\setbeamertemplate{footline}[text line]{}

\begin{document}
\prepareCover

\section{Course Administration}

\begin{frame}[fragile]
\frametitle{Course Administration}
Assignment 4 released. Due on April 16, 2015 at 17:30 PM.
\end{frame}

\begin{frame}[fragile]
  \frametitle{Overview}
  \begin{itemize}
    \item[] Abstraction
    \begin{itemize}
      \item[] Introduction
      \item[] Abstract Classes and Methods
    \end{itemize}
  \end{itemize}
\end{frame}

\plain{}{Abstraction}

\section{Introduction}

\begin{frame}[fragile]
  \frametitle{Abstraction}
  \begin{block}{How Do We Think?}
    \begin{quote}
      - Hi Sara! What's up?\\
      + Hey Mike! What's up?
    \end{quote}
  \end{block}
  \begin{block}{How Do We Really Think?}
    \begin{quote}
      What did you do today?
    \end{quote}
  \end{block}
\end{frame}

\begin{frame}[fragile]
  \frametitle{Abstraction}
  \begin{block}{Objective}
    Write a program \texttt{Geometry.java} in which you can create cricles and squares and get their area, perimeter, color and whether they are filled or not.
  \end{block}
\end{frame}

\begin{frame}[fragile]
  \frametitle{Abstraction}
  \begin{block}{UML Diagram for Class \texttt{Shape}}
  \begin{figure}
  \centering
    \begin{tikzpicture}
      \begin{class}[]{Shape}{0, 0}
        \attribute{- numShapes: int}
        \attribute{- filled: boolean}
        \attribute{- color: String}
        \operation{+ Shape()}
        \operation{+ getNumShape(): int}
        \operation{+ isFilled(): boolean}
        \operation{+ setFilled(boolean filled)}
        \operation{+ getColor(): String}
        \operation{+ setColor(String color)}
      \end{class}
    \end{tikzpicture}
  \end{figure}
  \end{block}
\end{frame}

\begin{frame}[fragile]
  \frametitle{Abstraction}
  \begin{block}{UML Diagram for Class \texttt{Circle}}
  \begin{figure}
  \centering
    \begin{tikzpicture}
      \begin{class}[]{Circle}{0, 0}
        \attribute{- numCircles: int}
        \attribute{- radius: double}
        \operation{+ Circle()}
        \operation{+ getPerimeter(): double}
        \operation{+ getArea(): double}
        \operation{+ getNumCircles(): int}
        \operation{+ getRadius(): double}
        \operation{+ setRadius(double radius)}
      \end{class}
    \end{tikzpicture}
  \end{figure}
  \end{block}
\end{frame}

\begin{frame}[fragile]
  \frametitle{Abstraction}
  \begin{block}{UML Diagram for Class \texttt{Square}}
  \begin{figure}
  \centering
    \begin{tikzpicture}
      \begin{class}[]{Square}{0, 0}
        \attribute{- numSquares: int}
        \attribute{- length: double}
        \operation{+ Square()}
        \operation{+ getPerimeter(): double}
        \operation{+ getArea(): double}
        \operation{+ getNumSquares(): int}
        \operation{+ getLength(): double}
        \operation{+ setLength(double radius)}
      \end{class}
    \end{tikzpicture}
  \end{figure}
  \end{block}
\end{frame}

\begin{frame}[fragile]
  \frametitle{Abstraction}
  \begin{block}{\texttt{Shape.java} (v2.0) (Part 1)}
    \begin{minted}[fontsize=\small,tabsize=8, linenos, firstnumber=1]{java}
public class Shape {
  // attributes and fields
  private static int numShapes = 0;
  private String color;
  private boolean filled;
  // constructors
  public Shape() {
    numShapes++;
  }
  // setter and getter for color
  public void setColor(String someColor) {
    this.color = someColor;
  }
  public String getColor() {
    return color;
  }
    \end{minted}
  \end{block}
\end{frame}

\begin{frame}[fragile]
  \frametitle{Abstraction}
  \begin{block}{\texttt{Shape.java} (v2.0) (Part 2)}
    \begin{minted}[fontsize=\small,tabsize=8, linenos, firstnumber=17]{java}
  // setter and getter for filled
  public void setFilled(boolean state) {
    this.filled = state;
  }
  public boolean isFilled() {
    return filled;
  }
  // setter and getter for numShapes
  public static int getNumShapes() {
    return numShapes;
  }
}
    \end{minted}
  \end{block}
\end{frame}

\begin{frame}[fragile]
  \frametitle{Abstraction}
  \begin{block}{\texttt{Circle.java} (v2.0) (Part 1)}
    \begin{minted}[fontsize=\small,tabsize=8, linenos, firstnumber=1]{java}
public class Circle extends Shape {
  // fields and attributes
  private static int numCircles = 0;
  private double radius;
  // methods
  public double getArea() {
    return Math.PI*Math.pow(radius,2);
  }
  public double getPerimeter() {
    return 2*Math.PI*radius;
  }
  // constructors
  public Circle(double someRadius) {
    radius = someRadius;
    numCircles++;
  }
    \end{minted}
  \end{block}
\end{frame}

\begin{frame}[fragile]
  \frametitle{Abstraction}
  \begin{block}{\texttt{Circle.java} (v2.0) (Part 2)}
    \begin{minted}[fontsize=\small,tabsize=8, linenos, firstnumber=17]{java}
  // setter and getter for radius
  public void setRadius(double someRadius) {
    this.radius = someRadius;
  }
  public double getRadius() {
    return radius;
  }
  // setter and getter for numCircles
  public static int getNumCircles() {
    return numCircles;
  }
}
    \end{minted}
  \end{block}
\end{frame}

\begin{frame}[fragile]
  \frametitle{Abstraction}
  \begin{block}{\texttt{Square.java} (v2.0) (Part 1)}
    \begin{minted}[fontsize=\small,tabsize=8, linenos, firstnumber=1]{java}
public class Square extends Shape {
  // fields and attributes
  private static int numSquares = 0;
  private double length;
  // methods
  public double getArea() {
    return Math.pow(length,2);
  }
  public double getPerimeter() {
    return length*4;
  }
  // constructors
  public Square(double someLength) {
    length = someLength;
    numSquares++;
  }
    \end{minted}
  \end{block}
\end{frame}

\begin{frame}[fragile]
  \frametitle{Abstraction}
  \begin{block}{\texttt{Square.java} (v2.0) (Part 2)}
    \begin{minted}[fontsize=\small,tabsize=8, linenos, firstnumber=17]{java}
  // setter and getter for length
  public void setLength(double someLength) {
    this.length = someLength;
  }
  public double getLength() {
    return length;
  }
  // setter and getter for numSquares
  public static int getNumSquares() {
    return numSquares;
  }
}
    \end{minted}
  \end{block}
\end{frame}

\begin{frame}[fragile]
  \frametitle{Abstraction}
  \begin{block}{\texttt{Geometry.java} (v2.0)}
    \begin{minted}[fontsize=\small,tabsize=8, linenos, firstnumber=1]{java}
public class Geometry {
  public static void main(String[] args) {
    Circle myCircle = new Circle(5);
    myCircle.setColor("Red");
    myCircle.setFilled(true);
    double result1 = myCircle.getPerimeter();
    System.out.println("Perimeter of circle: "+result1);
    Square mySquare = new Square(4);
    mySquare.setColor("Blue");
    mySquare.setFilled(false);
    double result2 = mySquare.getArea();
    System.out.println("Area of square: "+result2);
    System.out.println(Shape.getNumShapes());
  }
}
    \end{minted}
  \end{block}
\end{frame}

\begin{frame}[fragile]
  \frametitle{Abstraction}
  \begin{block}{\texttt{UML Diagram (v2.0)}}
  \begin{figure}
  \centering
    \begin{tikzpicture}
      \begin{class}[]{Shape}{0, 0}
        \attribute{numShapes: int}
        \attribute{filled: boolean}
        \attribute{color: String}
        \operation{Shape()}
      \end{class}
      \begin{class}[]{Circle}{-3, -3.25}
        \inherit{Shape}
        \attribute{numCircles: int}
        \attribute{radius: double}
        \operation{Circle()}
        \operation{getPerimeter(): double}
        \operation{getArea(): double}
      \end{class}
      \begin{class}[]{Square}{3, -3.25}
        \inherit{Shape}
        \attribute{numSquares: int}
        \attribute{length: double}
        \operation{Square()}
        \operation{getPerimeter(): double}
        \operation{getArea(): double}
      \end{class}
    \end{tikzpicture}
  \end{figure}
  \end{block}
\end{frame}

\section{Abstract Classes and Methods}

\begin{frame}[fragile]
  \frametitle{Abstraction}
  \begin{block}{Problem Statement}
    \begin{itemize}
      \item[] Superclass can be instantiated.
      \item[] Subclasses may not implement \texttt{getPerimeter()} and \texttt{getArea()} methods.
    \end{itemize}
    \begin{minted}[fontsize=\small,tabsize=8, linenos, firstnumber=1]{java}
public class Evil {
  public static void main(String[] args) {
    Shape myShape = new Shape();
    myShape.setColor("Red");
    myShape.setFilled(true);
  }
}
    \end{minted}
  \end{block}
\end{frame}

\begin{frame}[fragile]
  \frametitle{Abstraction}
  \begin{block}{Proposed Solution}
    Disallow instantiation of the \texttt{Shape} class.

    Force all subclasses of \texttt{Shape} class to implement \texttt{getArea()} and \texttt{getPerimeter()} methods.
  \end{block}
\end{frame}

\begin{frame}[fragile]
  \frametitle{Abstraction}
  \begin{block}{Definition}
    Abstraction is the act of hiding internal details of implementation of behaviors of object and providing functionalities of the object instead.
  \end{block}
\end{frame}

\begin{frame}[fragile]
  \frametitle{Abstraction}
  \begin{block}{Abstract Methods}
    Abstract methods define signature of methods that must be implemented in subclasses.
    \begin{minted}[fontsize=\small,tabsize=8, linenos, firstnumber=1]{java}
public abstract void getArea();
public abstract void getPerimeter();
    \end{minted}
  \end{block}
\end{frame}

\begin{frame}[fragile]
  \frametitle{Abstraction}
  \begin{block}{Abstract Classes}
    Classes declared as abstract may not be instantiated.

    Abstract classes may or may not have abstract methods.

    Classes with abstract methods must be declared as abstract.
  \end{block}
    \begin{minted}[fontsize=\small,tabsize=8, linenos, firstnumber=1]{java}
public abstract class Shape {
  public abstract double getArea();
  public abstract double getPerimeter();
  // other class members
}
    \end{minted}
\end{frame}

\begin{frame}[fragile]
  \frametitle{Abstraction}
  \begin{block}{Remember}
    Subclasses extending an abstract class must implement its abstract methods or be declared as abstract.
  \end{block}
    \begin{minted}[fontsize=\small,tabsize=8, linenos, firstnumber=1]{java}
public class Circle {
  private double radius;
  @Override
  public double getArea() {
    return Math.PI*Math.pow(radius,2);
  }
  @Override
  public double getPerimeter() {
    return this.radius*Math.PI*2;
  }
  // other class members
}
    \end{minted}
\end{frame}

\begin{frame}[fragile]
  \frametitle{Abstraction}
  \begin{block}{\texttt{Shape.java} (v3.0) (Part 1)}
    \begin{minted}[fontsize=\small,tabsize=8, linenos, firstnumber=1]{java}
public abstract class Shape {
  // fields
  private static int numShapes = 0;
  // attributes
  private String color;
  private boolean filled;
  // constructors
  public Shape() {
    numShapes++;
  }
  // abstract methods
  public abstract double getPerimeter();
  public abstract double getArea();
    \end{minted}
  \end{block}
\end{frame}

\begin{frame}[fragile]
  \frametitle{Abstraction}
  \begin{block}{\texttt{Shape.java} (v3.0) (Part 2)}
    \begin{minted}[fontsize=\small,tabsize=8, linenos, firstnumber=14]{java}
  // setters and getters
  public static int getNumShapes() {
    return numShapes;
  }
  public void setColor(String someColor) {
    this.color = someColor;
  }
  public String getColor() {
    return color;
  }
  public void setFilled(boolean state) {
    this.filled = state;
  }
  public boolean isFilled() {
    return filled;
  }
}
    \end{minted}
  \end{block}
\end{frame}

\begin{frame}[fragile]
  \frametitle{Abstraction}
  \begin{block}{\texttt{Circle.java} (v3.0) (Part 1)}
    \begin{minted}[fontsize=\small,tabsize=8, linenos, firstnumber=1]{java}
public class Circle extends Shape {
  // fields
  private static int numCircles = 0;
  // attributes
  private double radius;
  // methods
  @Override
  public double getArea() {
    return Math.PI*Math.pow(radius,2);
  }
  @Override
  public double getPerimeter() {
    return 2*Math.PI*radius;
  }
    \end{minted}
  \end{block}
\end{frame}

\begin{frame}[fragile]
  \frametitle{Abstraction}
  \begin{block}{\texttt{Circle.java} (v3.0) (Part 2)}
    \begin{minted}[fontsize=\small,tabsize=8, linenos, firstnumber=15]{java}
  // constructors
  public Circle(double someRadius) {
    radius = someRadius;
    numCircles++;
  }
  // setter and getter for radius
  public void setRadius(double someRadius) {
    this.radius = someRadius;
  }
  public double getRadius() {
    return radius;
  }
  // setter and getter for numCircles
  public static int getNumCircles() {
    return numCircles;
  }
}
    \end{minted}
  \end{block}
\end{frame}

\begin{frame}[fragile]
  \frametitle{Abstraction}
  \begin{block}{\texttt{Square.java} (v3.0) (Part 1)}
    \begin{minted}[fontsize=\small,tabsize=8, linenos, firstnumber=1]{java}
public class Square extends Shape {
  // fields and attributes
  private static int numSquares = 0;
  private double length;
  // methods
  @Override
  public double getArea() {
    return Math.pow(length,2);
  }
  @Override
  public double getPerimeter() {
    return length*4;
  }
    \end{minted}
  \end{block}
\end{frame}

\begin{frame}[fragile]
  \frametitle{Abstraction}
  \begin{block}{\texttt{Square.java} (v3.0) (Part 2)}
    \begin{minted}[fontsize=\small,tabsize=8, linenos, firstnumber=15]{java}
  // constructors
  public Square(double someLength) {
    length = someLength;
    numSquares++;
  }
  // setters and getters
  public void setLength(double someLength) {
    this.length = someLength;
  }
  public double getLength() {
    return length;
  }
  public static int getNumSquares() {
    return numSquares;
  }
}
    \end{minted}
  \end{block}
\end{frame}

\begin{frame}[fragile]
  \frametitle{Abstraction}
  \begin{block}{\texttt{Geometry.java} (v3.0)}
    \begin{minted}[fontsize=\small,tabsize=8, linenos, firstnumber=1]{java}
public class Geometry {
  public static void main(String[] args) {
    Circle myCircle = new Circle(5);
    myCircle.setColor("Red");
    myCircle.setFilled(true);
    double result1 = myCircle.getPerimeter();
    System.out.println("Perimeter of circle: "+result1);
    Square mySquare = new Square(4);
    mySquare.setColor("Blue");
    mySquare.setFilled(false);
    double result2 = mySquare.getArea();
    System.out.println("Area of square: "+result2);
    System.out.println(Shape.getNumShapes());
  }
}
    \end{minted}
  \end{block}
\end{frame}

\plain{}{Keep Calm\\and\\Think Object-Oriented}

\end{document}
