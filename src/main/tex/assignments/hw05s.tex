%%%%%%%%%%%%%%%%%%%%%%%%%%%%%%%%%%%%%%%%%%%%%%%%%%%%%%%%%%%%%%%%%%%%%%
% UMB-CS110-2015S: Introduction to Computing
% Copyright 2015 Pejman Ghorbanzade <pejman@ghorbanzade.com>
% Creative Commons Attribution-ShareAlike 4.0 International License
% More info: https://github.com/ghorbanzade/UMB-CS110-2015S
%%%%%%%%%%%%%%%%%%%%%%%%%%%%%%%%%%%%%%%%%%%%%%%%%%%%%%%%%%%%%%%%%%%%%%

\def \topDirectory {.}
\def \texDirectory {\topDirectory/src/main/tex}

\documentclass[12pt,letterpaper,twoside]{article}
\usepackage{\texDirectory/template/style/directives}
\usepackage{\texDirectory/template/style/assignment}
%%%%%%%%%%%%%%%%%%%%%%%%%%%%%%%%%%%%%%%%%%%%%%%%%%%%%%%%%%%%%%%%%%%%%%%%%%%%%%
% CS110: Introduction to Computing
% Copyright 2015 Pejman Ghorbanzade <mail@ghorbanzade.com>
% Creative Commons Attribution-ShareAlike 4.0 International License
% https://github.com/ghorbanzade/UMB-CS110-2015S/blob/master/LICENSE
%%%%%%%%%%%%%%%%%%%%%%%%%%%%%%%%%%%%%%%%%%%%%%%%%%%%%%%%%%%%%%%%%%%%%%%%%%%%%%

\course{id}{CS110}
\course{name}{Introduction to Computing}
\course{venue}{Tue/Thu, 5:30 PM - 6:45 PM}
\course{semester}{Spring 2015}
\course{department}{Department of Computer Science}
\course{university}{University of Massachusetts Boston}

\instructor{name}{Pejman Ghorbanzade}
\instructor{title}{}
\instructor{position}{Student Instructor}
\instructor{email}{pejman@cs.umb.edu}
\instructor{phone}{617-287-6419}
\instructor{office}{S-3-124B}
\instructor{office-hours}{Tue/Thu 19:00-20:30}
\instructor{address}{University of Massachusetts Boston, 100 Morrissey Blvd., Boston, MA}


\begin{document}

\doc{title}{Solution to Assignment 5}
\doc{date-pub}{Apr 19, 2015 at 00:00 AM}
\doc{date-due}{Apr 30, 2015 at 05:30 PM}
\doc{points}{8}

\prepare{header}

\section*{Question 1}

Write a program \texttt{FlightTest.java} that controls two airplanes, one \texttt{plane1} a commercial plane from \texttt{Eagle} company and one \texttt{plane2}, a fighter-bomber from \texttt{Dragon} company.  All airplanes fly at some altitude and with some speed. They take off and land. Just as bombers may drop bomb (if they have any) while they are in sky, commercials can board passengers (as much as their capacity allows) while grounded.

Assuming all planes made by \texttt{Eagle} have 300 seats, fly at 39000 ft and 550 mph and all planes made by \texttt{Dragon} have 2 bomb bays and fly at 30000 ft and 1000 mph, develop classes \texttt{Aircraft}, \texttt{Bomber}, \texttt{Commercial}, \texttt{Eagle} and \texttt{Dragon} and use them in a program \texttt{FlightTest.java}.

\section*{Solution}

\begin{enumerate}
\item Class \texttt{Aircraft.java}

\lstset{language=java,tabsize=2}
\begin{lstlisting}
public abstract class Aircraft {
	// attributes
	private double flyAltitude;
	private double flySpeed;
	private double altitude;
	private double speed;
	private boolean grounded;
	// methods
	public void takeOff() {
		this.grounded = false;
		this.altitude = this.flyAltitude;
		this.speed = this.flySpeed;
		System.out.println("Plane took off.");
	}
	public void land() {
		this.grounded = true;
		this.altitude = 0;
		this.speed = 0;
		System.out.println("Plane landed.");
	}
	// constructor
	public Aircraft(double altitude, double speed) {
		this.flyAltitude = altitude;
		this.flySpeed = speed;
		this.land();
		System.out.println("New Aircraft Made.");
	}
	// getters and setters
	public boolean isGrounded() {
		return grounded;
	}
	public double getAltitude() {
		return altitude;
	}
	public double getSpeed() {
		return speed;
	}
}
\end{lstlisting}

\item Class \texttt{Bomber.java}

\lstset{language=java,tabsize=2}
\begin{lstlisting}
public abstract class Bomber extends Aircraft {
	// attributes
	private int numBombs;
	// methods
	public void dropBomb() {
		if (!this.isGrounded() && numBombs > 0) {
			numBombs--;
			System.out.println("Bomb dropped.");
		}
	}
	// constructor
	public Bomber(double altitude, double speed, int bombs) {
		super(altitude, speed);
		this.numBombs = bombs;
		System.out.println("A new bomber!");
	}
	// getters and setters
	public int getNumBombs() {
		return numBombs;
	}
}
\end{lstlisting}

\item Class \texttt{Commercial.java}

\lstset{language=java,tabsize=2}
\begin{lstlisting}
public abstract class Commercial extends Aircraft {
	// attributes
	private int numPassenger;
	// methods
	public void board(int num) {
		if (this.isGrounded()) {
			numPassenger += num;
			System.out.println(num + " passenger(s) aboard.");
		}
	}
	// constructor
	public Commercial(double altitude, double speed) {
		super(altitude, speed);
		numPassenger = 0;
		System.out.println("A new commercial!");
	}
	// getters and setters
	public int getNumPassenger() {
		return numPassenger;
	}
}
\end{lstlisting}

\item Class \texttt{Eagle.java}

\lstset{language=java,tabsize=2}
\begin{lstlisting}
public class Eagle extends Commercial {
	// attributes
	private static int maxPassenger = 300;
	private static double altitude = 39000;
	private static double speed = 550;
	// methods
	public void board() {
		this.board(1);
	}
	public void board(int num) {
		if (this.getNumPassenger() + num < maxPassenger) {
			super.board(num);
		}
	}
	// constructor
	public Eagle() {
		super(altitude, speed);
		System.out.println("A new eagle!");
	}
	// getters and setters
}
\end{lstlisting}

\item Class \texttt{Dragon.java}

\lstset{language=java,tabsize=2}
\begin{lstlisting}
public class Dragon extends Bomber {
	// attributes
	private static int maxNumBombs = 2;
	private static double altitude = 30000;
	private static double speed = 1000;
	// methods
	// constructor
	public Dragon() {
		super(altitude, speed, maxNumBombs);
		System.out.println("A new dragon!");
	}
	// getters and setter
}
\end{lstlisting}

\item Class \texttt{FlightTest.java}

\lstset{language=java,tabsize=2}
\begin{lstlisting}
public class FlightTest {
	public static void main(String[] args) {
		// instantiate planes
		// For the commercial plane
		Eagle plane1 = new Eagle();
		// when plane is created, it's grounded
		// and boarding 10 passengers is possible
		plane1.board(10);
		// if the plane takes off
		plane1.takeOff();
		// we can't board passengers anymore
		plane1.board(10);
		// and there will only be 10 passengers aboard
		System.out.println("Num Passengers: "+plane1.getNumPassenger());
		// it flies in 39000 ft
		System.out.println("Altitude: "+plane1.getAltitude());
		// and its speed is 550 mph
		System.out.println("Speed: "+plane1.getSpeed());
		// if the plane lands again
		plane1.land();
		// its altitude becomes 0
		System.out.println("Altitude: "+plane1.getAltitude());
		// as well as its speed
		System.out.println("Speed: "+plane1.getSpeed());
		// As for the bomber plane
		Dragon plane2 = new Dragon();
		// when plane is created, it's grounded
		// and loaded with 2 bombs
		System.out.println("Num Bombs: "+plane2.getNumBombs());
		// but it cannot drop bomb while grounded
		plane2.dropBomb();
		// if the plane takes off
		plane2.takeOff();
		// it can drop bombs
		plane2.dropBomb();
		// until its bomb bays are empty
		plane2.dropBomb();
		// that's when it cannot drop bomb anymore
		plane2.dropBomb();
		// it flies at 30000 ft
		System.out.println("Altitude: "+plane2.getAltitude());
		// and its speed is 1000 mph
		System.out.println("Speed: "+plane2.getSpeed());
		// until it lands
		plane2.land();
		// its altitude becomes 0
		System.out.println("Altitude: "+plane1.getAltitude());
		// as well as its speed
		System.out.println("Speed: "+plane1.getSpeed());
	}
}
\end{lstlisting}

\end{enumerate}

\section*{Question 2}

Your manager has just learned basic concepts of interfaces and abstraction in Java and has asked you to implement the methods in the following interface in a class \texttt{ArraySort.java}. Write a program \texttt{ArraySortTest.java} to show him how to use your class to initialize, shuffle and sort an array. Attach a file \texttt{note.txt} to briefly explain why this is not a good practice and recommend a better alternative.

\lstset{language=java,tabsize=2}
\begin{lstlisting}
public interface sortAndUnsort {
	// sorts the array using insertion sort
	public int[] sortInsertion(int[] array);
	// shuffles the array
	public int[] shuffle(int[] array);
	// prints the array elements
	public void print(int[] array);
	// initialize array from firstElement to lastElement
	public int[] init(int firstElement, int lastElement);
}
\end{lstlisting}

\section*{Solution}

\begin{enumerate}
\item Class \texttt{ArraySort.java}

\lstset{language=java, tabsize=2}
\begin{lstlisting}
public class ArraySort implements sortAndUnsort {
	// sorts the array using insertion sort
	@Override
	public int[] sortInsertion(int[] array) {
		for (int i = 0; i < array.length; i++) {
			int minIndex = i;
			int min = array[i];
			for (int j = i; j < array.length; j++) {
				if (array[j] < min) {
					min = array[j];
					minIndex = j;
				}
			}
			if (i != minIndex) {
				int temp = array[i];
				array[i] = array[minIndex];
				array[minIndex] = temp;
			}
		}
		return array;
	}
	// shuffles the array
	@Override
	public int[] shuffle(int[] array) {
		for (int i = 0; i < array.length; i++) {
			int j = (int) (i + (array.length - i)*Math.random());
			int temp = array[j];
			array[j] = array[i];
			array[i] = temp;
		}
		return array;
	}
	// prints the array elements
	@Override
	public void print(int[] array) {
		for (int i = 0; i < array.length; i++) {
			System.out.printf("%2d ", array[i]);
		}
		System.out.println();
	}
	// initialize array from firstElement to lastElement
	@Override
	public int[] init(int firstElement, int lastElement) {
		int[] array = new int[lastElement - firstElement + 1];
		for (int i = 0; i < array.length; i++) {
			array[i] = firstElement + i;
		}
		return array;
	}
}
\end{lstlisting}

\item Class \texttt{ArraySortTest.java}

\lstset{language=java,tabsize=2}
\begin{lstlisting}
public class ArraySortTest {
	public static void main(String[] args) {
		ArraySort instance = new ArraySort();
		// initialize array from 1 to 10
		int[] array = instance.init(1, 10);
		// prints the initialized array
		instance.print(array);
		// shuffles the array
		array = instance.shuffle(array);
		// prints the shuffled array
		instance.print(array);
		// sorts array using insertion sort
		array = instance.sortInsertion(array);
		// prints the sorted array
		instance.print(array);
	}
}
\end{lstlisting}

The disadvantage of using an interface for the proposed objective is that given methods should be implemented in a separate class. Calling these methods require instantiation from \texttt{ArraySort} class. One possible alternative is to develop all desired methods as static methods of normal class \texttt{ArraySort.java} whose constructor is made private, eliminating the need for an interface. This way, the following code snippet can replace the previous \texttt{ArraySortTest.java} file.

\lstset{language=java,tabsize=2}
\begin{lstlisting}
public class ArraySortTestV2 {
	public static void main(String[] args) {
		// initialize array from 1 to 10
		int[] array = ArraySort.init(1, 10);
		// prints the initialized array
		ArraySort.print(array);
		// shuffles the array
		array = ArraySort.shuffle(array);
		// prints the shuffled array
		ArraySort.print(array);
		// sorts array using insertion sort
		array = ArraySort.sortInsertion(array);
		// prints the sorted array
		ArraySort.print(array);
	}
}
\end{lstlisting}

\end{enumerate}

\end{document}
