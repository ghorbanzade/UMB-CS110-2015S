\section*{Question 1}
According to National Weather Serivce, the \textit{Wind Chill} can be computed as given in Equation \ref{eq1}, where $t$ is the temperature in Fahrenheit and $v$ is the wind speed in miles per hour.
\begin{equation}
w = 35.74 + 0.6215 t + (0.4275t-35.75)v^{0.16}
\label{eq1}
\end{equation}
Write a program \texttt{WindChill.java} that takes temperature in Celsius and using Equation \ref{eq1} gives the wind chill in Celsius.
Fahrenheit to Celsius conversion formula is given in Equation \ref{eq2}.
\begin{equation}
t_F = 1.8 \times t_C + 32
\label{eq2}
\end{equation}
\begin{figure}[H]\centering
\includegraphics[width=8cm]{\topDirectory/template/images/windchill.png}
\end{figure}
\subsection*{Solution}
\lstset{language=Java}
\begin{lstlisting}
public class WindChill {
	public static void main (String[] args) {

		// Given command-line arguments are of String nature.
		// So we have to convert them to double type to perform our calculations.
		// Remember that first argument is args[0].
		double tempCelsius = Double.parseDouble(args[0]);
		double windVelocity = Double.parseDouble(args[1]);

		// Converting Temperature from Celsius to Fahrenheit
		double tempFahrenheit = 1.8 * tempCelsius + 32;

		// Calculating The Wind Chill In Fahrenheits
		double windChillFahrenheit = 35.74 + 0.6215*tempFahrenheit + (0.4275*tempFahrenheit - 5.75)*Math.pow(windVelocity,0.16);

		// Converting Wind Chill from Fahrenheit to Celsius
		double windChillCelsius = 9/5*(windChillFahrenheit - 32);
		int windChill = (int) Math.round(windChillCelsius);
		System.out.print("When temperature is " + tempCelsius + " Celsius and wind is " + windVelocity + " Miles per hour, ");
		System.out.println("Wind chill is " + windChill + " degrees Fahrenheit.");
	}
}
\end{lstlisting}
\section*{Question 2}
Write a program \textit{Maximum.java} that takes three integer numbers as command line arguments and prints their maximum, minimum and mean values.
\subsection*{Solution}
\lstset{language=Java}
\begin{lstlisting}
public class Maximum {
	public static void main (String[] args) {

		// Converting command-line arguments to integer numbers
		int num1 = Integer.parseInt(args[0]);
		int num2 = Integer.parseInt(args[1]);
		int num3 = Integer.parseInt(args[2]);

		// Maximum number
		// We used conditional ternary operations to make our code more concise, elegant and easy to read
		// The following conditional ternary operation is equivalent to
		// if (num1 > num2) { max = num1; } else {	max = num2; }
		int max = (num1 > num2) ? num1 : num2;
		max = (max > num3) ? max : num3;
		System.out.println("Maximum number is " + max);

		// Minimum number
		int min = (num1 > num2) ? num2 : num1;
		min = (min > num3) ? num3 : min;
		System.out.println("Minimum numer is " + min);

		// Mean Value
		double mean = (num1 + num2 + num3)/3;
		System.out.printf("Mean value is %.2f\n", mean);
	}
}
\end{lstlisting}
