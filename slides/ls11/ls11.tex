% -----------------------------------------------------------------------------
% The MIT License (MIT)
%
% Copyright (c) 2015 Pejman Ghorbanzade
%
% Permission is hereby granted, free of charge, to any person obtaining a copy
% of this software and associated documentation files (the "Software"), to deal
% in the Software without restriction, including without limitation the rights
% to use, copy, modify, merge, publish, distribute, sublicense, and/or sell
% copies of the Software, and to permit persons to whom the Software is
% furnished to do so, subject to the following conditions:
%
% The above copyright notice and this permission notice shall be included in
% all copies or substantial portions of the Software.
%
% THE SOFTWARE IS PROVIDED "AS IS", WITHOUT WARRANTY OF ANY KIND, EXPRESS OR
% IMPLIED, INCLUDING BUT NOT LIMITED TO THE WARRANTIES OF MERCHANTABILITY,
% FITNESS FOR A PARTICULAR PURPOSE AND NONINFRINGEMENT. IN NO EVENT SHALL THE
% AUTHORS OR COPYRIGHT HOLDERS BE LIABLE FOR ANY CLAIM, DAMAGES OR OTHER
% LIABILITY, WHETHER IN AN ACTION OF CONTRACT, TORT OR OTHERWISE, ARISING FROM,
% OUT OF OR IN CONNECTION WITH THE SOFTWARE OR THE USE OR OTHER DEALINGS IN
% THE SOFTWARE.
% -----------------------------------------------------------------------------

\def \topDirectory {../..}

\documentclass[10pt, compress]{beamer}

\usepackage{\topDirectory/template/style/directives}
%%%%%%%%%%%%%%%%%%%%%%%%%%%%%%%%%%%%%%%%%%%%%%%%%%%%%%%%%%%%%%%%%%%%%%%%%%%%%%
% CS110: Introduction to Computing
% Copyright 2015 Pejman Ghorbanzade <mail@ghorbanzade.com>
% Creative Commons Attribution-ShareAlike 4.0 International License
% https://github.com/ghorbanzade/UMB-CS110-2015S/blob/master/LICENSE
%%%%%%%%%%%%%%%%%%%%%%%%%%%%%%%%%%%%%%%%%%%%%%%%%%%%%%%%%%%%%%%%%%%%%%%%%%%%%%

\course{id}{CS110}
\course{name}{Introduction to Computing}
\course{venue}{Tue/Thu, 5:30 PM - 6:45 PM}
\course{semester}{Spring 2015}
\course{department}{Department of Computer Science}
\course{university}{University of Massachusetts Boston}

\instructor{name}{Pejman Ghorbanzade}
\instructor{title}{}
\instructor{position}{Student Instructor}
\instructor{email}{pejman@cs.umb.edu}
\instructor{phone}{617-287-6419}
\instructor{office}{S-3-124B}
\instructor{office-hours}{Tue/Thu 19:00-20:30}
\instructor{address}{University of Massachusetts Boston, 100 Morrissey Blvd., Boston, MA}

\usepackage{\topDirectory/template/style/beamerthemeUmassLecture}
\usepackage[school]{\topDirectory/template/style/pgf-umlcd}
\doc{number}{11}
%\setbeamertemplate{footline}[text line]{}

\begin{document}
\prepareCover

\section{Course Administration}

\begin{frame}[fragile]
\frametitle{Course Administration}
Assignment 4 released. Due on April 16, 2015 at 17:30 PM.
\end{frame}

\begin{frame}[fragile]
  \frametitle{Overview}
  \begin{itemize}
    \item[] Array of Objects
    \item[] Class Arrays
    \item[] Class ArrayList
  \end{itemize}
\end{frame}

\section{Array of Objects}

\begin{frame}[fragile]
  \frametitle{Array of Objects}
  \begin{block}{Review}
    An array is a simple data structure that holds a \textbf{fixed} number of elements of a \textbf{certain} data type.

    Each array element is identified by its \textbf{numerical index}.

    An array is stored based on position of its \textbf{first} element.

    First element of an array has index \textbf{0}.
  \end{block}
\end{frame}

\begin{frame}[fragile]
  \frametitle{Array of Objects}
  \begin{block}{\texttt{SimpleArray.java}}
    \begin{minted}[fontsize=\small,tabsize=8, linenos, firstnumber=1]{java}
public class SimpleArray {
  public static void main(String[] args) {
    boolean[][] seats = new boolean[5][10];
    for (int i = 0; i < seats.length; i++)
      for (int j = 0; j < seats[0].length; j++)
        seats[i][j] = false;
    seats[3][7] = true;
    for (int i = 0; i < seats.length; i++) {
      for (int j = 0; j < seats[0].length; j++)
        System.out.print(seats[i][j]+" ");
      System.out.println();
    }
  }
}
    \end{minted}
  \end{block}
\end{frame}

\begin{frame}[fragile]
  \frametitle{Array of Objects}
  \begin{block}{Objective}
    Write a program \texttt{Circles.java} that constructs an array of size 10 in whose \texttt{i}th index there is a \texttt{Circle} of radius \texttt{i}, and find perimeter of each Circle.
  \end{block}
\end{frame}

\begin{frame}[fragile]
  \frametitle{Array of Objects}
  \begin{block}{\texttt{Circle.java}}
    \begin{minted}[fontsize=\small,tabsize=8, linenos, firstnumber=1]{java}
public class Circle {
  private int radius;
  public double getPerimeter() {
    return 2 * Math.PI * this.radius;
  }
  public Circle(int someRadius) {
    radius = someRadius;
  }
  public int getRadius() {
    return radius;
  }
}
    \end{minted}
  \end{block}
\end{frame}

\begin{frame}[fragile]
  \frametitle{Array of Objects}
  \begin{block}{\texttt{Circles.java}}
    \begin{minted}[fontsize=\small,tabsize=8, linenos, firstnumber=1]{java}
public class Circles {
  public static void main(String[] args) {
    Circle[] array = new Circle[10];
    for (int i = 0; i < array.length; i++) {
      array[i] = new Circle(i);
    }
    for (int i = 0; i < array.length; i++) {
      System.out.println(array[i].getPerimeter());
    }
  }
}
    \end{minted}
  \end{block}
\end{frame}

\section{Class Arrays}

\begin{frame}[fragile]
  \frametitle{Class Arrays}
  \begin{block}{Objective}
    Write a program \texttt{arraySort.java} that asks for elements of an array of size 10 and sorts all elements in ascending order.
  \end{block}
\end{frame}

\begin{frame}[fragile]
  \frametitle{Class Arrays}
  \begin{block}{\texttt{arraySort.java} (v1.0) (Part 1)}
    \begin{minted}[fontsize=\small,tabsize=8, linenos, firstnumber=1]{java}
import java.util.Scanner;
public class arraySort {
  private static int[] array;
  public static void main(String[] args) {
    array = new int[10];
    getElements();// getting elements from user
    sortArray(array);// sort the array
    printArray(array);// print sorted array
  }//end of main method
    \end{minted}
  \end{block}
\end{frame}

\begin{frame}[fragile]
  \frametitle{Class Arrays}
  \begin{block}{\texttt{arraySort.java} (v1.0) (Part 2)}
    \begin{minted}[fontsize=\small,tabsize=8, linenos, firstnumber=1]{java}
  private static void printArray(int[] array) {
    for (int i = 0; i < array.length; i++)
      System.out.printf("%2d, ",array[i]);
  }// end of printArray method
  private static void getElements() {
    Scanner input = new Scanner(System.in);
    for (int i = 0; i < array.length; i++) {
      System.out.printf("Element in index %2d: ", i);
      array[i] = input.nextInt();
    }
    input.close();
  }//end of getElements method
    \end{minted}
  \end{block}
\end{frame}

\begin{frame}[fragile]
  \frametitle{Class Arrays}
  \begin{block}{\texttt{arraySort.java} (v1.0) (Part 3)}
    \begin{minted}[fontsize=\small,tabsize=8, linenos, firstnumber=1]{java}
  private static void sortArray(int[] array) {
    //insertion sort
    for (int i = 0; i < array.length; i++)
      for (int j = i; j < array.length; j++)
        if (array[j] < array[i])
          array = swap(array, i, j);
  }//end of sortArray method
  public static int[] swap(int[] array, int i, int j) {
    int temp = array[i];
    array[i] = array[j];
    array[j] = temp;
    return array;
  }//end of swap method
}

    \end{minted}
  \end{block}
\end{frame}

\begin{frame}[fragile]
  \frametitle{Class Arrays}
  \begin{block}{Problem Statement}
    Sorting is very common in almost any application and is widely taken advantage of.
  \end{block}
  \begin{block}{Proposed Solution}
    Use a pre-developed class that provides methods for commonly-used operations on arrays.
  \end{block}
\end{frame}

\begin{frame}[fragile]
  \frametitle{Class Arrays}
  \begin{block}{\texttt{arraySort.java} (v1.0) (Part 1)}
    \begin{minted}[fontsize=\small,tabsize=8, linenos, firstnumber=1]{java}
package arrays;
import java.util.Arrays;
import java.util.Scanner;
public class arraySort2 {
  private static int[] array;
  public static void main(String[] args) {
    array = new int[10];
    getElements();// getting elements from user
    Arrays.sort(array);// sort the array
    printArray(array);// print sorted array
  }//end of main method
    \end{minted}
  \end{block}
\end{frame}

\begin{frame}[fragile]
  \frametitle{Class Arrays}
  \begin{block}{\texttt{arraySort.java} (v1.0) (Part 2)}
    \begin{minted}[fontsize=\small,tabsize=8, linenos, firstnumber=1]{java}
  private static void printArray(int[] array) {
    for (int i = 0; i < array.length; i++) {
      System.out.printf("%2d, ",array[i]);
    }
  }// end of printArray method
  private static void getElements() {
    Scanner input = new Scanner(System.in);
    for (int i = 0; i < array.length; i++) {
      System.out.printf("Element in index %2d: ", i);
      array[i] = input.nextInt();
    }
    input.close();
  }//end of getElements method
}
    \end{minted}
  \end{block}
\end{frame}

\begin{frame}[fragile]
  \frametitle{Class Arrays}
  \begin{block}{Formal Introduction}
    the Class \texttt{java.util.Arrays} provides static computing-efficient methods for accessing and manipulating arrays.
  \end{block}
  \begin{block}{Methods}
    \begin{columns}
      \begin{column}{0.5\textwidth}
        \begin{itemize}
          \item[] binarySearch()
          \item[] copyOf()
          \item[] copyOfRange()
          \item[] equals()
          \item[] deepEquals()
        \end{itemize}
      \end{column}
      \begin{column}{0.5\textwidth}
        \begin{itemize}
          \item[] setAll()
          \item[] fill()
          \item[] hashCode()
          \item[] sort()
          \item[] parallelSort()
        \end{itemize}
      \end{column}
    \end{columns}
  \end{block}
\end{frame}

\section{Class ArrayList}

\begin{frame}[fragile]
  \frametitle{Class ArrayList}
  \begin{block}{Remember}
    Arrays hold a \textbf{fixed} number of elements of a \textbf{certain} data type.
  \end{block}
  \begin{block}{Problem Statement}
    Once declared, size of an array cannot be modified.
  \end{block}
  \begin{block}{Proposed Solution}
    \texttt{List}: A more advanced data structure with no limitation on size.
  \end{block}
\end{frame}

\begin{frame}[fragile]
  \frametitle{Class ArrayList}
  \begin{block}{List Data Structure}
    Lists are ordered collections with possibly duplicate elements.
  \end{block}
  \begin{block}{Java Implementation}
    \texttt{java.util.List<E>} is a generic \alert{interface} that is implemented by following widely-used classes.
    \begin{columns}
      \begin{column}{0.5\textwidth}
        \begin{itemize}
          \item[] AbstractList
          \item[] AbstractSequentialList
          \item[] ArrayList
          \item[] AttributeList
          \item[] CopyOnWriteArrayList
        \end{itemize}
      \end{column}
      \begin{column}{0.5\textwidth}
        \begin{itemize}
          \item[] LinkedList
          \item[] RoleList
          \item[] RoleUnresolvedList
          \item[] Stack
          \item[] Vector
        \end{itemize}
      \end{column}
    \end{columns}
  \end{block}
\end{frame}

\begin{frame}[fragile]
  \frametitle{Class ArrayList}
  \begin{block}{Class ArrayList}
    ArrayList is a special implementation of List that realizes dynamically-resizable arrays and provides computing-efficient methods for maintaining them.
  \end{block}
\end{frame}

\begin{frame}[fragile]
  \frametitle{Class ArrayList}
  \begin{block}{Methods}
    \begin{columns}
      \begin{column}{0.5\textwidth}
        \begin{itemize}
          \item[] add()
          \item[] remove()
          \item[] size()
          \item[] clear()
          \item[] clone()
          \item[] contains()
        \end{itemize}
      \end{column}
      \begin{column}{0.5\textwidth}
        \begin{itemize}
          \item[] ensureCapacity()
          \item[] get()
          \item[] indexOf()
          \item[] lastIndexOf()
          \item[] remove()
          \item[] trimToSize()
        \end{itemize}
      \end{column}
    \end{columns}
  \end{block}
\end{frame}

\begin{frame}[fragile]
  \frametitle{Class ArrayList}
  \begin{block}{\texttt{ArrayListTest.java}}
    \begin{minted}[fontsize=\small,tabsize=8, linenos, firstnumber=1]{java}
import java.util.ArrayList;
public class arraySort3 {
  public static void main(String[] args) {
    ArrayList<Integer> array = new ArrayList<Integer>(2);
    array.add(9);
    array.add(5);
    array.add(8);
    System.out.println(array + " size: "+array.size());
    array.remove(1);
    System.out.println(array);
    array.remove(array.indexOf(8));
    System.out.println(array);
  }
}
    \end{minted}
  \end{block}
\end{frame}

\plain{}{Keep Calm\\and\\Think Object-Oriented}

\end{document}
